\documentclass[12pt]{article}

\usepackage{amssymb,amsmath,amsthm}
\usepackage{graphicx} % Package for including figures
%\usepackage{psfrag,color}

\theoremstyle{definition}
\newtheorem{thm}{Theorem}[section]
\newtheorem{lem}[thm]{Lema}
\newtheorem{prop}[thm]{Proposition}
\newtheorem*{cor}{Corrolary}

\theoremstyle{definition}
\newtheorem{defn}{Definition}[section]
\newtheorem{conj}{Conjecture}[section]
\newtheorem{exmp}{Example}[section]


\title{Report: Homework 3 Math/CS 471}
\author{Teo Brandt and Brennan Collins}
\date{\today}   % Activate to display a given date or no date


\begin{document}
\maketitle

\begin{abstract}
This report will explore two methods of approximating the following integral
\[
I=\int_{-1}^{1}e^{cos(kx)}dx,
\]
for \(k=pi\text{ or }pi^{2}\). The first method is known as the trapezoidal rule and the second as Gauss quadrature.
\end{abstract}

\section{Trapezoidal Rule}
The trapezoidal rule is given by the following expression
\[
\int_{X_{L}}^{X_R}f(x)dx\approx h\bigg(\frac{f(x_{0})+f(x_{n})}{2}+\sum_{i=1}^{n-1}f(x_{i})\bigg)
\]
where the grid is given by \(x_{i}=X_{L}+ih\text{, }i=0,...,n\text{, }h=\frac{X_{R}-X_{L}}{n}\).
\section{Gauss Quadrature}
\cite{HW}In Gauss quadrature the location of the grid-points and weights, \(\omega_{i}\), are chosen so that the order of the approximation to the weighted integral
\[
\int_{-1}^{1}f(z)w(z)dz\approx \sum_{i=0}^{n}\omega_{i}f(z_{i}),
\]
is maximized. (The function \(w(z)\) is positive and integrable. In this report we will only consider the case when \(w(z)=1\) in order to simplify things.



\section{Methods}
Here is how the programs were executed...
\section{Results}
\begin{figure}[h]
\caption{Plot of error against \textbf{n}}
\centering
\includegraphics
{error_plot}
\end{figure}
\\
In the figure shown above, different rates of convergence are observed for each of the methods and for each of the values of \(k\). The trapezoidal method where \(k=\pi^{2}\) is the only case in which the order of the method may be read from the slope of its plot. This slope is \(\approx -3\) which is consistent with the theory as shown for
\newpage
\section{Appendix}
\indent In order to compile and execute the code for this assignment perl is used. The directory in which the code can be found is:
\\
\begin{center}
\textit{~/Homework/Homework2/Code/}
\end{center}
\\
Once in this directory the following command will compile and execute the code:
\\
\begin{center}
\textit{\$ perl newtonS.p}
\end{center}
\\

\newpage
\begin{thebibliography}{9}
\bibitem{HW} 
Daniel Appelo
\textit{Homework 3}. 
referenced Sep. 26, 2015

\end{thebibliography}

\end{document} 